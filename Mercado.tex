\documentclass[10pt, spanish, a5paper]{article}
\usepackage[spanish]{babel}
\selectlanguage{spanish}
\usepackage[utf8]{inputenc}

\usepackage{tgschola}
%\usepackage{tgheros}
%\renewcommand*\familydefault{\sfdefault}
%\usepackage{ascii}
\usepackage{cite}
%\usepackage[bookmarks]{hyperref}
\usepackage{tikz}
\usepackage{forest}
\usepackage{microtype}
\usepackage[margin=15mm]{geometry}
\usepackage{booktabs}
\usepackage{enumitem}
\setlist[description]{style=nextline}

\begin{document}
\title{\textbf{Mercadotecnia}}

\maketitle


\tableofcontents



\newpage



\part{Administración}
Es el proceso de \emph{fijar objetivos} y la \emph{orientación} de una organización.

También es el conjunto de funciones que los administradores deben llevar a cabo. Las funciones son:
\begin{itemize}
	\item Planificar.
	\item Controlar.
	\item Definir objetivos.
	\item Tomar decisiones en caso de problemas.
	\item Comunicar.
	\item Ejercer influencia y/o poder.
	\item Capacitar al personal.
\end{itemize}

\section{Administración (con mayúsculas)}
\begin{description}
	\item[Arte.]	porque se ejercita en base a a la \emph{experiencia transmitida} y  se perfecciona en la \emph{vida cotidiana.}
\item[Ciencia.] porque contribuye a un \emph{sistema de conocimientos metódicamente fundados} cuyo objetivo de estudio son las \emph{organizaciones}.
\item[Técnica.] porque tiene un \emph{conjunto de herramientas, principios, normas y procedimientos} aplicables a la conducción de las organizaciones.
\end{description}

\section{Funciones de las administración}

\begin{description}
	\item[Planificación.] Selección de \emph{misiones} y \emph{objetivos} y las \emph{acciones necesarias para lograrlos}.
	\item[Organización.] Establece una \emph{estructura intencional de roles} que deben cumplir las personas en una organización.
	\item[Integración del personal.] Incluye \emph{cubrir} y \emph{mantener cubiertos} los \emph{puestos} en la estructura de la organización.
	\item[Dirección.] \emph{Influenciar} a las personas para que \emph{contribuyan a las metas} de la organización y el grupo.
	\item[Control.] \emph{Medición} y \emph{corrección} de las actividades de los empleados para asegurar que los acontecimientos estén de acuerdo con lo planificado.
\end{description}

\part{Gestión}
Es la \emph{evaluación} de los \emph{resultados} de las \emph{acciones} y \emph{decisiones} de un administrador.


\section{Niveles de gestión de una organización}
\subsection{Estratégico}
Es el que determina cuáles son los negocios en los que una organización debe participar y cómo.

\subsection{Coordinador}
Son las \emph{gerencias que apoyan al nivel estratégico]}

Los gerentes son los encargados de 
\begin{itemize}
	\item Tomar decisiones
	\item Planificar
	\item Organizar
	\item Dirigir
	\item Controlar
\end{itemize}

\subsection{Operativo}

Es el que lleva a cabo las tareas y actividades para el cumplimiento de las metas.

\begin{figure}[h]
\centering
\begin{tabular}{ll}
	\midrule \textbf{Estratégico} & Jefe \\ 
	\midrule \textbf{Coordinador} & Gerentes \\ 
	\midrule \textbf{Operativo} & Empleados \\ 
	\midrule 
\end{tabular}
\caption{Los niveles de gestión en una organización}
\end{figure}

\section{Habilidades necesarias para cada nivel de gestión}

\begin{description}
	\item[Hablidad técnica.] Saber hacer.
	
	Saber trabajar con ciertos \emph{métodos}, \emph{procedimientos}, \emph{herramientas} y \emph{técnicas específicas}.
	
	\item[Habilidad humana.] Saber relacionarse.
	
	Es la capacidad de \emph{trabajar con personas}, es el \emph{esfuerzo cooperativo}, el \emph{trabajo en equipo}, la creación de un medio en el cual se sientan seguros y libres de expresar sus opiniones.

	\item[Habilidad conceptual.] Es la capacidad de \emph{ver la gran imagen}, \emph{reconocer los elementos importantes} en una situación y \emph{comprender las relaciones entre los elementos}.
\end{description}


\part{Planeamiento estratégico}

\section{Visión}
	\begin{itemize}
		\item Determinación de la empresa.
		\item Explica cómo quiere que sea en el futuro:
		\begin{itemize}
			\item Cultura.
			\item Recursos.
			\item Estrategia.
			\item Mercados.
			\item Organizaciones.
		\end{itemize}
	\end{itemize}

\section{Misión}
\begin{itemize}
	\item Orientación principal de una organización.
	\item Representación.
\end{itemize}

\section{Objetivos}
\begin{itemize}
	\item Son la definición concreta de la visión y misión.
	\item Los objetivos tienen:
	\begin{itemize}
		\item Atributo (verbo).
		\item Unidad de medida (porcentaje, tiempo).
		\item Cantidad (a cuánto se aspira).
		\item Horizonte de tiempo (lapso para llegar al objetivo).
	\end{itemize}
\end{itemize}

\section{Estrategia}

Es el camino que deciden recorrer los directivos para lograr los objetivos.

\subsection{Tipos de estrategias \cite{ansoff}}



\begin{description}
	\item[Penetración de mercado.] Que los actuales clientes compren más.
	
	Las principales estrategias son:
	\begin{itemize}
	\item Aumento del consumo o ventas de los clientes/usuarios actuales.
	\item Captación de clientes de la competencia.
	\item Captación de no consumidores actuales.
	\item Atraer nuevos clientes del mismo segmento aumentando publicidad y/o promoción.
	\end{itemize}
	
	\item[Desarrollo de mercado] Innovar. Crear o modificar nuevos segmentos de mercado.
	
	Las principales estrategias son:
	\begin{itemize}
	\item Apertura de mercados geográficos adicionales.
	\item Atracción de otros sectores del mercado.
	\item Política de distribución y posicionamiento
	\item Investigación y cambio del segmento.
	\end{itemize}
	
	\textsc{\textbf{ejemplo:}}\textit{ Lanzamiento del iPad. Cuando salió sólo habían netbooks y notebooks. Crearon un mercado nuevo de dispositivos que quedaban en medio de los teléfonos y las notebooks.}
	
	\item[Desarrollo de productos.] Busca que los clientes compren algo distinto.

Las principales estrategias son:
\begin{itemize}
	\item Desarrollo de nuevos valores del producto.
	\item Desarrollo de diferencias de calidad (nuevas gamas).
	\item Desarrollo de nuevos modelos o tamaños.
	\item Producto calidad.
\end{itemize}

\item[Diversificación] Productos nuevos para clientes nuevos.

	\textsc{\textbf{ejemplo:}}\textit{ Virgin.}
\end{description}

\part{Organizaciones}

Una organización es una \emph{manera particular del hombre de asociarse} para alcanzar una \emph{solución} a ciertos \emph{problemas} que se plantean.

Son importantes porque:
\begin{itemize}
	\item Las sociedad actual es una \emph{sociedad organizacional}.
	\item \emph{Ejercen poder} en la sociedad y \emph{modelan la vida humana}.
	\item Inducen a la \emph{especialización} y la \emph{profesionalización}
\end{itemize}

\section{Ejemplos de organizaciones}
\begin{itemize}
	\item Hospitales
	\item Cárceles
	\item Escuelas
	\item Ejércitos
	\item Parroquias
	\item Partidos políticos
	\item Empresas de servicio
\end{itemize}

\section{Organizaciones según el tipo de estructura}

\subsection{Organización formal}
Es una \emph{estructura intencional de roles} en una empresa formalmente organizada.

\subsection{Organización informal}
Es una \emph{red de relaciones personales y sociales no establecidas} que se producen \emph{espontáneamente} cuando las personas se asocian entre si.

\section{Características de una organización}

\subsection{División del trabajo}

Existen \emph{diferentes áreas} con \emph{diferentes funciones}. Por ejemplo: gerencia de personal, de comercialización y a su vez, se subdivide en departamentos.

\subsection{División del poder}

\emph{Cada miembro} tendrá \emph{diferente poder}, y en los \emph{niveles superiores está concentrado}.

\subsection{División de responsabilidades en las comunicaciones}

Deriva de los anteriores y determina el \emph{nivel de información} y de \emph{mensajes} que se manejan en cada nivel de la estructura.

\subsection{Preferencia de uno o más controles de poder}

Presencia de \emph{grupos que dirigen los esfuerzos hacia los fines}.

\subsection{Sustitución de personal}

Las personas que \emph{no satisfacen} lo que se espera de ella pueden ser \emph{sustituidas}.

\section{Recursos de las organizaciones}

Los recursos con los que cuenta y necesitan las organizaciones para desarrollar sus actividades y lograr sus fines son:

\subsection{Recursos humanos}

\begin{itemize}
	\item Personal de la organización (empleados).
	\item Los dueños de la organización.
	\item Cantidad de personal de tiempo completo.
\end{itemize}

\subsection{Recursos materiales}

\begin{itemize}
	\item Materias primas.
	\item Inmuebles.
	\item Máquinas.
	\item Muebles.
	\item Financieros.
\end{itemize}

\subsection{Recursos naturales y energéticos}

\subsection{Ideas, conocimiento e información}

\subsection{Recursos tecnológicos}

\subsection{Recursos intangibles}

\begin{itemize}
	\item Nombre.
	\item Prestigio.
	\item Símbolos.
	\item Marca.
\end{itemize}


\part{Modelo Penta de Alberto Levy}

El modelo PENTA es una \emph{herramienta técnica} para diagnosticar e intervernir en las empresas para ayudarlas a crear valor económico.

\section{Estrategia}
Son los propósitos de la organización.

\section{Recursos}

\begin{itemize}
	\item La gente. \textit{Recursos humanos.}
	\item Tangibles
	\item Intangibles
		\begin{itemize}
			\item Operacionales.\\
			\textit{Plantas, materia prima, etc.}
			\item Financieros.\\
			\textit{Capital}.
			\item Infraestructura. \\
			\textit{Depósitos, oficinas, IT, etc.}
		\end{itemize}
\end{itemize}

\section{Mercados}

El valor de un mercado depende su estrategia, cultura, recursos y sus organizaciones.

\section{Cultura}
Valores, creencias y aspiraciones de la empresa.

\section{Organización}
\begin{itemize}
	\item Organigrama.
	\item Sistemas de información.
	\item Procesos gerenciales.
\end{itemize}


\part{Empresas}

Es una unidad social que tiene por objetivos \emph{producir bienes y servicios que satisfagan necesidades de una comunidad}.

\section{Tipos de empresa}



\begin{figure*}[t]
	\centering
	{\footnotesize 
		
		
		\begin{tabular*}{\linewidth}{p{.33\linewidth} p{.33\linewidth} p{.33\linewidth}}
			\toprule[1pt]
			\textbf{Dimensional} & \textbf{Jurídicos}                           & \textbf{Actividad}  \\ \midrule
			Pequeñas             & Unipersonales                                & Industriales        \\ \midrule
			Medianas             & Sociedades colectivas o de personas          & Servicio industrial \\ \midrule
			Grandes              & Sociedades en comanditas                     & Comerciales         \\ \midrule
			& Sociedades de responsabilidad Limitada (SRL) &  \\ \midrule
			& Sociedades cooperativas                      &  \\ \midrule
			& Sociedades anónimas                          &  \\ \midrule
		\end{tabular*}
		
		
		\bigskip
		
		
		
		\begin{tabular*}{\linewidth}{p{.33\linewidth} p{.33\linewidth} p{.33\linewidth}}
			\toprule[1pt]
			\textbf{Económico} & \textbf{Geográfico} & \textbf{Propiedad} \\ \midrule
			Sector primario    & Locales             & Privadas           \\ \midrule
			Sector secundario  & Regionales          & Públicas           \\ \midrule
			Sector terciario   & Nacionales          &  \\ \midrule
		\end{tabular*}
		
	}
	
	
	
	\caption[Tipos de empresa]{Clasificación de los distintos tipos de empresa según distintos criterios}
	\label{fig:tiposempresa}
\end{figure*}



\subsection{Criterio económico}

\begin{description}
		\item[Sector primario.] Extracción de un recurso. \textit{Agricultura, ganadería, minería.}
		\item[Sector secundario.] Manufactura. Elabora productos de consumo final o partes de otras empresas manufactureras.
		\item[Sector terciario.] Servicios. \textit{Transporte, educación, salud, telecomunicaciones, etc.}
\end{description}

\subsection{Criterio geográfico}

\begin{itemize}
	\item Locales.
	\item Regionales.
	\item Nacionales.
\end{itemize}


\subsection{Régimen de propiedad}

\begin{description}
	\item[Privadas] Su capital está en manos de uno o más particulares (personas físicas o jurídicas).
	
	Su finalidad es el lucro.

	\item[Públicas] El capital está en manos del Estado.
	
		Su fin no es el lucro sino el servicio a la comunidad.
\end{description}

\subsection{Criterio dimensional}

\begin{description}
	\item[Pequeños] Capitales reducidos.
	
	Dimensión física y operaciones de magnitud limitada.
	\item[Medianas] 
	\item[Grandes] Elevados capitales invertidos.
	
	Tienen plantas, locales, oficinas, y equipos de gran magnitud.
	
	Muchos empleados
\end{description}


\subsection{Criterio jurídico}

\begin{description}
	\item[Unipersonal.] Un solo dueño.
	\item[Sociedades colectivas o de personas.] Todos los socios administran por si o por un mandatario de común acuerdo.

	Responden de forma \emph{indefinida} con sus \emph{propios bienes}.
	
\end{description}

\subsection{Sociedades en comanditas}
Tiene dos tipos de socios:
\begin{enumerate}
	\item \textbf{Socios comanditarios}
	\begin{itemize}
		\item Aportan el capital.
		\item Responden \emph{sólo por el monto de su aporte}.
	\end{itemize}
	\item \textbf{Socios gestores}
	\begin{itemize}
		\item Aportan el trabajo.
		\item Administran la empresa.
		\item Responden de manera ilimitada.
	\end{itemize}
\end{enumerate}

\subsection{Sociedades de Responsabilidad Limitada (SRL)}

\begin{itemize}
	\item Similar a la sociedad de personas
	\item Responsabilidad \emph{limitada} al monto de sus aportantes, salvaguardando sus bienes personales.
\end{itemize}

\subsection{Sociedades cooperativas}

\begin{itemize}
	\item Los socios se unen por un \emph{propósito en común}.
	\item La finalidad es mejorar las condiciones socio-económicas de sus socios.
	\item No tiene fines de lucro.
\end{itemize}


\subsection{Sociedades Anónimas (SA)}
\begin{itemize}
	\item Muchos socios.
	\item Responsabilidad limitada al monto de su aporte (expresado en número de acciones).
\end{itemize}



\part{Mercadotecnia}

\section{Definición de mercadotecnia}
Es un \emph{conjunto de técnicas} que busca producir bienes y servicios que \emph{satisfagan las necesidades} del \emph{consumidor} y \emph{generen valor} para la empresa.

\section{Evolución de la mercadotecnia}

\subsection{1800: Orientación a la producción}

\begin{itemize}
	\item La demanda era mayor que la oferta.
	\item La población crecía rápidamente.
	\item Se buscaba aumentar la producción para poder suplir las necesidades.
\end{itemize}

\subsection{Principios de los 50: Orientación a la venta}

\begin{itemize}
	\item Oferta $\neq$ Demanda
	\item Ofrecer un buen producto ya no era garantía de que te lo compren.
	\item El público tenía muchos opciones.
	\item Comenzaron a usar actividades promocionales para poder vender lo que la empresa quería fabricar.
\end{itemize}

\subsection{Mediados de los 50 (posguerra): Orientación a la satisfacción de necesidades}

\begin{itemize}
	\item Durante la posguerra la demanda de bienes de consumo era enorme.
	\item Las fábricas producían cantidades extraordinarias de bienes
	\item Los consumidores eran más difíciles de convencer.
	\item Las empresas comenzaron a identificar necesidades y dirigir sus recursos a satisfacerlas.
	
\end{itemize}



\section{Necesidades}

Es la carencia de algo unida al deseo de satisfacerla.\footnote{En psicología la necesidad es el sentimiento ligado a la vivencia de una carencia, lo que se asocia al esfuerzo orientado a suprimir esta falta, a satisfacer la tendencia, a la corrección de la situación de carencia.\cite{necesidad}}

\subsection{Necesidades básicas}

Ropa, alimento, seguridad, vivienda.

\subsection{Necesidades sociales}
Pertenencia, afecto.

\subsection{Necesidades individuales}
Conocimiento, expresión del yo.

\section{Deseos}
Son la forma que le da la cultura y la personalidad individual a las necesidades básicas.

Sed $\Rightarrow$ agua = Necesidad
Sed $\Rightarrow$ Coca Cola = Deseo

\section{Demanda}

Cuando el poder adquisitivo respalda los deseos, estos se convierten en demandas.

\subsection{Tipos de demanda}

\begin{description}
	\item[Demanda primaria] Demanda de productos y 
	servicios para el consumo final.
	
	\item[Demanda derivada] Es cuando un bien o servicio trae acarreado otro bien o servicio.
	
	\item[Demanda elástica] Varía el precio y varía la demanda.
	
	\textbf{\textsc{ejemplo}:} Los televisores de plasma.
	
	\item[Demanda genérica] Demanda global para una clase de producto.
	
	\item[Demanda inelástica] Varía el precio pero no la demanda.
	
	\textbf{\textsc{ejemplo}:} Luz, agua, higiene femenina.
	
	\item[Demanda potencial] Posibles consumidores que tienen algún interés por algún producto o servicio en particular.

	\textbf{\textsc{ejemplo}:}Alibaba y el mercado B2B.
	
	\item[Demanda negativa] Los consumidores pagan para evitar un producto o servicio que les desagrada.
	
	\item[Demanda ausente] Los consumidores-meta no tienen interés en el producto/servicio.

	\item[Demanda latente] Los consumidores tienen una necesidad que ningún producto o servicio existente la satisface.
	
	\item[Demanda irregular] La demanda varía según la temporada, los días, horas, etc.
	\textbf{\textsc{ejemplo}:} Helados, pantalla solar, fotografía de bodas, etc.
	
	\item[Demanda plena] La organización tiene la cantidad de demanda que quiere y puede manejar.

	\textbf{\textsc{ejemplo}:} Alfajores <<Capitán del espacio>> (Buenos Aires).
	
	\item[Demanda saturada] La organización tiene una demanda superior a la que desea y puede manejar.
	
	
\end{description}


\section{Administración mercadotécnica}

Existen cinco visiones alternativas según las cuales las organizaciones desarrollan sus actividades mercadotécnicas:

\subsection{Concepto de producción}

\begin{itemize}
	\item Los consumidores optarán por el producto más barato.
	\item Apunta a mejorar la eficacia de la \emph{producción} y de la \emph{distribución}.
\end{itemize}

\subsection{Concepto de producto}
Los consumidores prefieren productos que ofrezcan:
\begin{itemize}
	\item Calidad
	\item Rendimiento
	\item Renovación
\end{itemize}

\subsection{Concepto de Ventas}

Los consumidores no compran en cantidad productos de una organización a menos que esta realice ventas a gran nivel y promociones a gran escala.

\subsection{Concepto de Mercadotecnia}

La organización debe conocer las necesidades de los mercados-meta y satisfacerlas entregando valor, calidad y satisfacción a cambio de dinero.

\subsection{Mercadotecnia social}
\begin{quote}
<<Es la aplicación de las técnicas de la mercadotecnia comercial para el \emph{análisis}, \emph{planteamiento}, \emph{ejecución} y \emph{evaluación} de programas diseñados para influir en el comportamiento voluntario de la audiencia objetivo en orden a mejorar su bienestar personal y el de su sociedad.>>	
\end{quote}

\begin{itemize}
	\item La organización debe determinar las necesidades, anhelos e intereses de los mercados-meta.

	\item Analizan al consumidor inserto en una sociedad.
	
\end{itemize}

\section{Administración de las actividades mercadotécnicas}

\begin{enumerate}
	\item \textbf{Análisis.}
	\item \textbf{Planificación.} Elaborar planes estratégicos.
	\item \textbf{Aplicación.} Poner en práctica los planes.
	\item \textbf{Control.} Medir y evaluar resultados para tomar medidas correctas.
\end{enumerate}

\section{Objetivos de la mercadotecnia}
\begin{description}
	\item [Maximizar el consumo.] Aumentar la producción, el empleo y la riqueza.
	\item [Maximizar la satisfacción de los consumidores.] Adquirir un bien nuevo y lujoso, sirve sólo si aumenta la satisfacción del consumidor.
	\item [Maximizar las opciones.] Aumentar la variedad.
	\item [Maximizar la calidad de vida.] Mejorar la calidad del ambiente físico y cultura.
\end{description}

\section{Actividades mercadotécnicas}

\begin{itemize}
	\item Desarrollo de productos.
	\item Investigaciones.
	\item Comunicaciones.
	\item Distribución.
	\item Precios.
	\item Servicios.
\end{itemize}

\section{Desarrollo de productos}

\begin{enumerate}
	\item \textbf{Generación de ideas.}
	\item \textbf{Selección de ideas.}
	\item \textbf{Análisis comercial.} La idea seleccionada anteriormente se convierte en una propuesta concreta de negocio.
	\item \textbf{Creación del prototipo.} Al ser favorable el análisis, se elabora un prototipo.
	\item \textbf{Prueba de mercado.} Se prueba el producto con usuarios reales.
	\item \textbf{Comercialización.} Debe decidirse cuándo, dónde a quién y cómo introducir el nuevo producto en los mercados.
\end{enumerate}



\section{Marketing Mix}

Son las cuatro variables de decisión sobre las cuales las organizaciones tienen mayor control.
Las <<cuatro P>> son:

\begin{enumerate}
	\item \textbf{Producto.}
	
	Es el paquete total de beneficios que recibe el consumidor cuando compra.
	\item \textbf{Precio.}
	
	Es el costo financiero que representa para el cliente. Este incluye su distribución, descuentos, garantías, rebajas, etc.

	Para los consumidores potenciales, el precio es una expresión del valor del producto.
	
	\item \textbf{Distribución.}
	
	Se refiere a los intermediarios a través de los cuales llega el producto al consumidor. Por ejemplo: mayoristas, minoristas, distribuidores, agentes, etc.
	
	\item \textbf{Promoción.}
	
	Son los medios por los que se <<habla>> con los intermediarios, consumidores actuales y potenciales.
	
\end{enumerate}




\begin{figure*}[t]
	\centering
	{\footnotesize 
				
		\begin{tabular*}{\linewidth}{p{.2\linewidth} p{.2\linewidth} p{.2\linewidth} p{.3\linewidth}}
			\toprule[1pt]
			\textbf{Producto} & \textbf{Precio} & \textbf{Distribución} & \textbf{Promoción}  \\ \midrule
			Calidad           & Descuentos      & Canales               & Publicidad          \\ \midrule
			Características   & Listas          & Cubrimientos          & Ventas personales   \\ \midrule
			Estilos           & Plazos          & Lugares               & Promociones         \\ \midrule
			Marcas            & Intereses       & Inventarios           & Exhibiciones        \\ \midrule
			Empaque           & Niveles         & Transporte            & Ventas electrónicas \\ \midrule
			Tamaño            & Márgenes        & Almacenamiento        & Anuncios            \\ \midrule
			Garantía          & Condiciones     & Despacho              &  \\ \midrule
			Servicio          &                 &                       &  \\ \midrule
			Devoluciones      &                 &                       &  \\ \midrule
		\end{tabular*}
		
	}
	
	
	
	\caption[Las cuatro P]{Las cuatro P del marketing mix.}
	\label{fig:tiposempresa}
\end{figure*}
































\newpage

\bibliography{Mercado}
\bibliographystyle{acm}



\end{document}

